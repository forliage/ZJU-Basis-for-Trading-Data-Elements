\section{关键定理与证明 (Key Theorems and Proofs)}

我们将严谨地证明,在预算约束下,简洁的咨询机制 (Consulting Mechanism) 能够实现与任何复杂交互协议相同的最优收益。证明的核心是利用线性规划的对偶理论 (Duality Theory),通过一系列精巧的变换,揭示最优机制的内在结构。

\subsection{CM-probR定理}

我们将证明的核心定理是 [Chen et al. 2020] 的主要结果,即概率性返还的咨询机制 (CM-probR) 的最优性。这个证明过程也将隐含地证明其他简化情况下的最优性。

我们的证明将遵循之前提到的"三步走"策略:

\begin{itemize}
    \item \textbf{基准确立}:证明最优机制存在于存款-返还的定价结果机制 (POM-depR) 类别中。
    \item \textbf{对偶变换}:将寻找最优 POM-depR 的问题表述为原始LP $P$,然后通过$P\to D\to D'$的变换,得到一个变量数可控的对偶LP $D'$。
    \item \textbf{结构揭示}:将$D'$对偶回新的原始LP $P'$,并证明$P'$的最优解对应一个 CM-probR 机制。
\end{itemize}

\textbf{引理 5.1 (POM-depR 的最优性)}:对于任何一个能为卖方带来期望收益$R$的通用交互协议,都存在一个 POM-depR 机制,它能带来至少为$R$的期望收益。

现在,我们将 POM-depR 的最优化问题转化为一个线性规划,并对其进行变换。为使符号简洁,我们主要讨论公开预算$b$的情况。私有预算的情况可以通过将$(\theta,b)$视为一个复合类型来处理,分析是类似的。

如前所述,一个信号披露策略可以被等价地描述为对买方先验信念的凸分解 (convex decomposition)。

\textbf{外部观察者 (Outside Observer)}:假设有一个虚拟的外部观察者,他不知道买方的类型$\theta$,他看到的先验是$\mu(\omega)$。

\textbf{买方}:类型为$\theta$的买方,他看到的先验是$\mu(\omega|\theta)$。

\textbf{信念转换}:当外部观察者看到后验为$q$时,类型$\theta$的买方根据贝叶斯法则,看到的后验将是$D_\theta q / (1^T D_\theta q)$,其中$D_\theta$是一个对角矩阵,其对角线元素为$\mu(\theta|\omega)$。

\textbf{买方价值函数}:类型$\theta$的买方对外部后验$q$的(未归一化的)价值为$v_\theta(D_\theta q)$,其中$v_\theta(p)=\max_{a\in A}\sum\limits_{\omega}p(\omega)u(\theta,\omega,a)$。$v_{\theta}$是齐次函数,即
$$v_\theta(c \cdot p) = c \cdot v_\theta(p)\text{ for }c>0$$

一个 POM-depR 机制由一组$(x_\theta,t_\theta)$决定,其中$x_\theta$是一个后验分布,$t_\theta(q)$是对应后验$q$的价格。我们使用变量$x_\theta(q)$表示选择后验$q$的概率,$\tau_\theta(q) = x_\theta(q) t_\theta(q)$表示期望支付。

最优 POM-depR 的问题可以写成如下(可能无限维的)线性规划$P$:

\begin{align*}
    \max \quad & \sum_{\theta \in \Theta} \sum_{q \in Q^*} \tau_{\theta}(q) \\
    \text{s.t.} \quad & \sum_{q} (v_{\theta}(D_{\theta}q)x_{\theta}(q) - \tau_{\theta}(q)) - \sum_{q} (v_{\theta}(D_{\theta'}q)x_{\theta'}(q) - \tau_{\theta'}(q)) \geq 0 && \forall \theta, \theta' \quad && (\text{IC}) \quad : \lambda_{\theta, \theta'} \\
    & \sum_{q} (v_{\theta}(D_{\theta}q)x_{\theta}(q) - \tau_{\theta}(q)) \geq v_{\theta}(D_{\theta}\mu) && \forall \theta \quad && (\text{IR}) \quad : \alpha_{\theta} \\
    & \sum_{q} x_{\theta}(q) \cdot q = \mu && \forall \theta \quad && (\text{Feasibility}) \quad : y_{\theta} \\
    & b \cdot x_{\theta}(q) - \tau_{\theta}(q) \geq 0 && \forall \theta, q \quad && (\text{Budget}) \quad : \beta_{\theta}(q) \\
    & \tau_{\theta}(q) + M \cdot x_{\theta}(q) \geq 0 && \forall \theta, q \quad && (\text{Seller Budget}) \quad : \xi_{\theta}(q) \\
    & x_{\theta}(q) \geq 0, \beta_{\theta}(q) \geq 0, \xi_{\theta}(q) \geq 0
\end{align*}

这里$\mu$是$\mu(\omega)$的向量形式。$Q^*$是一个足够大的离散后验集合。在约束的右侧,我们标注了其对应的对偶变量。

我们来推导$P$的对偶问题$D$。$D$的目标是最小化原始问题约束右侧项与对偶变量乘积的和。$D$的约束来自于对每个原始变量($x_\theta(q)$和$\tau_\theta(q)$)的系数进行整理。

对于原始变量$\tau_\theta(q)$,其系数必须满足:
$$-\sum\limits_{\theta' \neq \theta}\lambda_{\theta,\theta'} \cdot (-1) - \alpha_{\theta}\cdot (-1)-\sum\limits_{\theta'\neq\theta}\lambda_{\theta',\theta}\cdot 1 - \beta_{\theta}(q)\cdot (-1) + \xi_{\theta}(q) \cdot 1 \geq 1$$

整理得:
$$1+\sum\limits_{\theta'}\lambda_{\theta',\theta}-\sum\limits_{\theta'}\lambda_{\theta,\theta'}-\alpha_{\theta}-\beta_{\theta}(q) - \xi_{\theta}(q) = 0\quad(\text{因为}P\text{是最大化问题})$$

令$\Lambda_\theta = 1 - \alpha_\theta - \sum\limits_{\theta' \neq\theta}(\lambda_{\theta,\theta'}-\lambda_{\theta',\theta})$,则有$\beta_{\theta}(q) + \xi_{\theta}(q) = \Lambda_\theta$。由于$\beta$和$\xi$非负,这意味着$\Lambda_\theta$必须非负,且$\beta$和$\xi$在$\Lambda_\theta$上进行分配。

对于原始变量$x_\theta(q)$,其系数必须满足:
$$\left(v_\theta(D_\theta q)(1 - \alpha_\theta - \sum\limits_{\theta' \neq\theta}\lambda_{\theta,\theta'}) - \sum\limits_{\theta' \neq \theta}v_{\theta'}(D_{\theta} q)\lambda_{\theta',\theta}\right) + (y_{\theta}^T q) + b\cdot \beta_\theta(q) - M\cdot \xi_{\theta}(q)\geq 0$$
(这里我们暂时忽略$\lambda_{\theta,\theta}$)。

将$\beta_\theta(q) = \Lambda_\theta - \xi_\theta(q)$带入,并重新整理,得到$D$的约束:
$$y_{\theta}^T q + v_\theta(D_\theta q)\Lambda_{\theta}^{\text{self}} - \sum\limits_{\theta'\neq\theta}v_{\theta'}(D_\theta q)\lambda_{\theta',\theta} + b\Lambda_\theta - (b+M)\xi_\theta(q) \geq 0 \quad \forall \theta,q$$
其中$\Lambda_\theta$和$\Lambda_{\theta}^{\text{self}}$是关于$\alpha,\lambda$的线性表达式。

对偶问题$D$如下:

\begin{align*}
    \min \quad & \sum_{\theta} v_{\theta}(D_{\theta}\mu)\alpha_{\theta} + y_{\theta}^{T}\mu \\
    \text{s.t.} \quad & y_{\theta}^{T}q + \cdots - (b+M)\xi_{\theta}(q) \geq 0 && \forall \theta, q \\
    & \Lambda_{\theta} \geq \xi_{\theta}(q) \geq 0 && \forall \theta, q \\
    & \alpha_{\theta}, \lambda_{\theta, \theta'} \geq 0
\end{align*}

这个对偶问题$D$仍然有指数多的变量$\xi_\theta(q)$。

\textbf{引理5.2}:对偶问题$D$的约束可以被等价地替换为一组新的约束,这组新的约束中不再含有变量$\xi_\theta(q)$。

我们来分析$D$的约束。对于固定的$\theta,\lambda,\alpha,y$,约束可以写成:

同时我们还有约束$0\leq \xi_\theta(q)\leq \Lambda_\theta$。为了使所有约束都成立,我们必须能够为每个$q$找到一个$\xi_\theta(q)$满足上述两个条件。这等价于要求两个上界中较小的一个,必须大于等于下界0。即,对于所有的$q$:
$$\min \left(\Lambda_\theta, \frac{y_{\theta}^T q+\cdots}{b+M}\right)\geq 0$$
由于$b+M>0$和$\Lambda_\theta\geq 0$,这等价于两条独立的约束:

\begin{enumerate}
    \item $\Lambda_\theta\geq 0$(这条已经隐含在$\xi_\theta(q)\leq \Lambda_\theta$和$\xi_\theta(q)\geq 0$中)
    \item $y_{\theta}^T q + \text{terms\_not\_depending\_on\_xi}(q)\geq 0$
\end{enumerate}

现在,让我们回到$D$约束的原始形式,并观察$\xi_\theta(q)$是如何出现的:
$$y_{\theta}^T q + \text{terms}(\lambda,\alpha)-(b+M)\xi_{\theta}(q) \geq 0$$

对于给定的$\lambda$、$\alpha$、$y$,是否存在$\beta_\theta(q)$、$\xi_\theta(q)$使得$D$的所有约束都满足。

$$\beta_\theta(q) + \xi_\theta(q) = \Lambda_\theta,\quad \beta_\theta(q)\geq 0, \xi_\theta(q)\geq 0$$

将$\beta_\theta(q)=\Lambda_\theta - \xi_\theta(q)$代入。

我们还需要满足$0\leq \xi_\theta(q) \leq \Lambda_\theta$。

因此,对于任意$q$,上述不等式的右侧必须有一个解$\xi_\theta(q)$存在于$[0,\Lambda_\theta]$。(假设$b+M>0$)

这给出了两条关于$y,\Lambda,\alpha$的,与$\xi$无关的约束:

\begin{enumerate}
    \item $y_{\theta}^T q - \text{cost\_term}(q) + b \Lambda_\theta \geq 0$
    \item $y_{\theta}^T q - \text{cost\_term}(q) - M \Lambda_\theta \leq 0$
\end{enumerate}

将$\text{cost\_term}(q)$和$\Lambda_\theta$展开回$\lambda,\alpha$的表达式,我们就得到了一个新的对偶LP $D'$。

\begin{align*}
    \min \quad & \sum_{\theta} v_{\theta}(D_{\theta}\mu)\alpha_{\theta} + y_{\theta}^{T}\mu \\
    \text{s.t.} \quad & y_{\theta}^{T}q - \left(v_{\theta}(D_{\theta}q)\Lambda_{\theta}^{\text{self}} - \sum_{\theta' \neq \theta} v_{\theta'}(D_{\theta}q)\lambda_{\theta', \theta}\right) + b\left(1 - \alpha_{\theta} - \sum_{\theta'} (\lambda_{\theta, \theta'} - \lambda_{\theta', \theta})\right) \geq 0 && \forall \theta, q \quad : x_{\theta}^{+}(q) \\
    & y_{\theta}^{T}q - \left(v_{\theta}(D_{\theta}q)\Lambda_{\theta}^{\text{self}} - \sum_{\theta' \neq \theta} v_{\theta'}(D_{\theta}q)\lambda_{\theta', \theta}\right) - M\left(1 - \alpha_{\theta} - \sum_{\theta'} (\lambda_{\theta, \theta'} - \lambda_{\theta', \theta})\right) \leq 0 && \forall \theta, q \quad : x_{\theta}^{-}(q) \\
    & \alpha_{\theta}, \lambda_{\theta, \theta'} \geq 0, y_{\theta} \in \mathbb{R}^{|\Omega|}
\end{align*}

这个$D'$只有多项式数量的变量$(\alpha,\lambda,y)$,但仍然有指数/无限多的约束(对每个$q$)。

现在,我们对$D'$进行对偶操作,得到新的原始LP $P'$。$P'$的变量是$x_{\theta}^{+}(q)$和$x_{\theta}^{-}(q)$,对应$D'$的两组主要约束。

对$D'$中每个对偶变量$\alpha_\theta$、$\lambda_{\theta,\theta'}$、$y_\theta$的系数进行整理,可以推导出$P'$的约束。$P'$的目标函数则来自$D'$约束的右侧项(在这里是$\mu$和$v_\theta(D_\theta \mu)$)。经过繁琐但直接的代数运算,我们得到$P'$的形式:

\begin{align*}
    \max \sum_{\theta, q} \quad & \left( b \cdot x_{\theta}^{+}(q) - M \cdot x_{\theta}^{-}(q) \right) \\
    \text{s.t.} \quad & \sum_{q} \left((v_{\theta}(D_{\theta}q) - b)x_{\theta}^{+}(q) + (v_{\theta}(D_{\theta}q) + M)x_{\theta}^{-}(q)\right) \\
    & - \sum_{q} \left((v_{\theta}(D_{\theta}q) - b)x_{\theta'}^{+}(q) + (v_{\theta}(D_{\theta}q) + M)x_{\theta'}^{-}(q)\right) \geq 0 && \forall \theta, \theta' \quad (\text{IC}) \\
    & \sum_{q} \left((v_{\theta}(D_{\theta}q) - b)x_{\theta}^{+}(q) + (v_{\theta}(D_{\theta}q) + M)x_{\theta}^{-}(q)\right) \geq v_{\theta}(D_{\theta}\mu) && \forall \theta \quad (\text{IR}) \\
    & \sum_{q} (x_{\theta}^{+}(q) + x_{\theta}^{-}(q)) \cdot q = \mu && \forall \theta \quad (\text{Feasibility}) \\
    & x_{\theta}^{+}(q) \geq 0, x_{\theta}^{-}(q) \geq 0
\end{align*}

$P'$的目标函数和约束完美地诠释了一个机制:
\begin{itemize}
    \item 当买方报告$\theta$时,以总概率$\sum\limits_{q} x_{\theta}^{+}(q)$要求支付净额$b$。
    \item 当买方报告$\theta$时,以总概率$\sum\limits_{q} x_{\theta}^{-}(q)$要求支付净额$-M$。
    \item 信息披露策略由后验分布$\{x_{\theta}^{+}(q), x_{\theta}^{-}(q)\}$描述。
\end{itemize}

我们需要证明$P'$存在一个最优解,其中信息披露等价于行动推荐。

\textbf{引理5.3}:线性规划$P'$存在一个最优解$\{x_{\theta}^{+},x_{\theta}^{-}\}$,使得对于任意$\theta$和任意支付结果$o \in \{+,-\}$,集合$\{q|x_{\theta}^{o}(q) > 0\}$中每一个$q$都对应一个不同的、由$\arg \max_{a}v_{\theta}(D_{\theta} q)$决定的最优行动。

假设我们有一个最优解$x^*$不满足这个性质。即,存在$\theta,o$和两个不同的后验$q_1,q_2$,使得$x_{\theta}^{o*}(q_1)>0, x_{\theta}^{o*}(q_2)>0$,并且:
$$\arg \max_{a} v_{\theta}(D_{\theta}q_1)=\arg\max_{a}v_{\theta}(D_{\theta}q_2)=a^*$$

现在,我们构造一个新的解$x_{\text{new}}$。令$w_1=x_{\theta}^{o*}(q_1)$和$w_2=x_{\theta}^{o*}(q_2)$。定义一个新的后验$q_{\text{new}} = \frac{w_1 q_1 + w_2 q_2}{w_1 + w_2}$。在$x_{\text{new}}$中,我们将$q_1$和$q_2$的概率清零,并将它们的总概率$w_1 + w_2$赋给$q_{\text{new}}$:

\begin{itemize}
    \item $x_{\theta,\text{new}}^{o}(q_1) = 0$
    \item $x_{\theta,\text{new}}^{o}(q_2) = 0$
    \item $x_{\theta,\text{new}}^{o}(q_{\text{new}}) = w_1 + w_2$
    \item 所有其它的$x$值保持不变。
\end{itemize}

我们需要验证$x_{\text{new}}$仍然是$P'$的一个可行解,并且目标函数值不变。

\begin{enumerate}
    \item \textbf{目标函数值}:目标函数只依赖于$x$的总和,与$q$的具体分布无关,因此目标值不变。
    \item \textbf{可行性约束}:$$\sum\limits_{q}\left(x_{\theta}^{+}(q)+x_{\theta}^{-}(q)\right)\cdot q = \mu$$ 新解在这一项的改变量是$(w_1 + w_2) q_{\text{new}} - (w_1 q_1 + w_2 q_2) = 0$。所以可行性约束仍然满足。
    \item \textbf{IC/IR 约束}:约束的每一项都是形如$\sum\limits_{q}f(q)x(q)$的求和。我们需要证明,这个合并操作不会使得任何约束的左侧项变得更小(从而可能违反$\geq$约束)。核心在于$v_\theta(D_\theta q)$函数的性质。因为$a^*$同时是$q_1$和$q_2$的最优行动,这意味着$v_\theta(D_\theta q)$在$q_1$和$q_2$之间的连线上是线性的。$$v_{\theta}(D_{\theta} q_{\text{new}}) = v_{\theta} \left(D_{\theta} \frac{w_1 q_1 + w_2 q_2}{w_1 + w_2}\right) = \frac{w_1 v_{\theta}(D_{\theta} q_1) + w_2 v_{\theta}(D_{\theta} q_2)}{w_1 + w_2}$$ 这个线性关系保证了所有约束项的值在合并前后保持不变。对于其他类型的买方$\theta' \neq \theta$,$v_{\theta'}(D_{\theta'} q)$不一定在$q_1$和$q_2$之间是线性的,但它一定是凸的。$$v_{\theta'}(D_{\theta'} q_{\text{new}})\leq \frac{w_1 v_{\theta'}(D_{\theta'}q_1) + w_2 v_{\theta'}(D_{\theta'}q_2)}{w_1 + w_2}$$ 这个凸性关系意味着,对于其他买方的IC约束(即当他们谎报成$\theta$时),合并后的新解给他们带来的效用不会增加,因此如果原约束满足,新约束也一定满足。
\end{enumerate}

通过这个构造性的"合并"证明,我们表明总能找到一个最优解,其中每个后验都对应一个唯一的行动推荐。此时,披露后验$q$等价于推荐行动$a$。这正是 CM-probR 机制的定义。

由于强对偶性,$P'$的最优值等于$P$的最优值。我们证明了最优的 POM-depR 的收益,可以被一个 CM-probR 机制达到。再结合引理 5.1,我们最终得出结论:CM-probR 是所有可能机制中的最优机制。

证毕!

\subsection{机制的稳健性:最优收益的连续性分析}

在前面的部分,我们都假设上下文$(u,\mu)$是精确已知且固定的。然而在实践中,卖方对联合分布$\mu$的估计可能存在误差。一个自然且重要的问题是:如果真实的上下文$(u',\mu')$与卖方所设想的$(u,\mu)$非常接近,那么最优收益$R(u',\mu')$是否也与$R(u,\mu)$相近?换言之,最优收益函数$R$是否是连续的?

在最一般的情况下(允许负支付),最优收益函数$R(u,\mu)$不是连续的,但它是下半连续的 (Lower Semicontinuous)。如果我们对机制施加一些"合理"的限制,比如不允许卖方向买方支付(即无正转移),或者机制是更简单的定价映射机制,那么最优收益函数就是连续的 (Continuous)。

我们将重点证明其中最核心、最微妙的部分:最优收益函数$R(u,\mu)$的下半连续性。

设$C=\mathbb{R}^{|\Theta|\times |\Omega|\times |A|}\times \Delta(\Theta\times \Omega)$为所有可能上下文$(u,\mu)$构成的空间,赋予其标准的欧几里得拓扑。最优收益函数$R:C\to \mathbb{R}$。

一个函数$f:X\to \mathbb{R}$在点$x$是下半连续的,如果对于任意收敛到$x$的点列$\{x_n\}$,都有:
$$f(x)\leq \lim\limits_{n\to \infty}\inf f(x_n)$$
直观上讲,这意味着$x_n$趋近于$x$时,函数值$f(x_n)$不会发生"向下的突然跳跃"。它可以"向上跳",但不能"向下跳"。

我们的目标是证明:对于任意上下文$c=(u,\mu)$,以及任意收敛到$c$的上下文序列$c_n = (u_n,\mu_n)$,我们有$R(u,\mu)\leq \lim \inf_{n\to \infty} R(u_n,\mu_n)$。

证明的关键在于,我们需要展示对于$c$的最优(或近似最优)机制,当上下文从$c$稍微扰动到$c_n$时,这个机制仍然是"近似可行的",并且其收益也只是发生了微小的变化。

根据前面的分析,我们知道最优机制存在于 POM-depR 类别中。一个定价结果机制可以被一个元组$M=(\{x_{\theta}\},\{\tau_{\theta}\})$所完全描述,其中:$x_\theta$是一个后验空间$\Delta(\Omega)$上的概率测度 (probability measure),满足$$\int_{q\in \Delta(\Omega)}q \, dx_{\theta}(q) = \mu$$(这里的$\mu$指$\mu(\omega)$分布)。$\tau_\theta$是一个$\Delta(\Omega)$上的带号测度 (signed measure),代表支付。

我们将机制空间$\mathcal{M}$看作是所有满足可行性、IC和IR约束的这种测度元组的集合。这里的后验空间$\Delta(\Omega)$是一个紧致的度量空间。测度空间(如$x_\theta$所在的$M(\Delta(\Omega))$)的拓扑结构需要小心处理。我们赋予它弱-*拓扑。

在泛函分析中,对于一个紧致空间$K$,其上的概率测度空间$P(K)$在弱-*拓扑下也是紧致的(根据 Banach-Alaoglu 定理)。弱-*拓扑下的收敛$\mu_n\to\mu$意味着,对于所有连续函数$f:K\to\mathbb{R}$,都有:
$$\int_K f \, d\mu_n\to \int_K f \, d\mu$$
这个拓扑结构非常适合我们的问题,因为所有的效用和支付的期望值,都可以表示为对某个连续函数(如$v_\theta(D_\theta q)$)的积分。

直接处理带等号的 IC/IR 约束很麻烦,因为当上下文$c$发生微小变化时,原本成立的等式可能就不成立了。我们需要一个更强的结论:

\textbf{引理5.4(严格偏好机制)}:对于任意上下文$c=(u,\mu)$和任意$\epsilon >0$,存在一个定价结果机制$M^*$,它能产生收益$R(M^*)\geq (1-\epsilon)R(u,\mu)$,并且在该机制下,所有的 IC 和 IR 约束都以严格不等式成立(除非两个合约完全相同)。

假设我们有一个最优机制$M_{\text{opt}}$,其收益为$R(u,\mu)$。考虑一个新的机制$M^*$,它的信息披露策略$x^*_{\theta}$与$M_{\text{opt}}$完全相同,但其支付$\tau_{\theta}^*(q)$是原支付的$(1-\epsilon)$倍,即$\tau^*_{\theta}(q) = (1 - \epsilon)\tau_{\theta}(q)$。

新机制的收益是$(1-\epsilon)R(u,\mu)$。

让我们检查IC约束:
$$\sum\limits_{q}(v_\theta(D_{\theta}q)x_\theta(q) - (1 - \epsilon)\tau_\theta(q))\geq \sum\limits_{q}\left(v_\theta(D_{\theta'}q)x_{\theta'}(q) - (1 - \epsilon)\tau_{\theta'}(q)\right)$$

将$(1-\epsilon)$提出,这等价于:
$$\sum\limits_{q}v_\theta(D_{\theta}q)x_{\theta}(q)-\sum\limits_{q}v_{\theta}(D_{\theta'}q)x_{\theta'}(q)\geq (1 - \epsilon)\left(\sum\limits_{q}\tau_{\theta}(q) - \sum\limits_{q}\tau_{\theta'}(q)\right)$$

而在原始机制$M_{\text{opt}}$中,IC 约束要求:
$$\sum\limits_{q} v_{\theta}(D_{\theta}q)x_{\theta}(q)-\sum\limits_{q}v_{\theta}(D_{\theta'}q)x_{\theta'}(q)\geq \sum\limits_{q}\tau_{\theta}(q)-\sum\limits_{q}\tau_{\theta'}(q)$$

比较这两条,如果原始约束是严格大于,新约束在$\epsilon$足够小时也成立。如果原始约束是严格等于,那么只有当$\sum \tau_{\theta}>0$时,新约束才可能被违反。但可以证明,这种情况可以通过调整消除。最终,我们可以构造出一个所有约束都严格成立的机制,其收益任意接近最优收益。

现在我们来完成下半连续性的证明。令$c_n = (u_n,\mu_n)$是一个收敛到$c=(u,\mu)$的上下文序列。我们的目标是证明$R(c)\leq \lim\inf R(c_n)$。

根据引理 5.4,对于任意给定的$\epsilon > 0$,我们可以在上下文$c$下找到一个机制$M_{\epsilon}$,它的收益$R(M_\epsilon)\geq (1-\epsilon)R(c)$,并且其所有的IC和IR约束都是严格大于。

一个机制$M=(\{x_\theta\},\{\tau_\theta\})$的可行性(即 IC、IR、Feasibility 约束的满足)可以被看作是一个对应 (correspondence) $F:C\rightrightarrows \mathcal{M}$,它将每个上下文$c$映射到所有在该上下文下可行的机制集合$F(c)$。
\begin{itemize}
    \item IC 约束形如:$$\int v_{\theta}(D_{\theta}q)\,dx_{\theta} - \int d\tau_{\theta} - \left(\int v_{\theta}(D_{\theta'}q)\,dx_{\theta'} - \int d\tau_{\theta'}\right)\geq 0$$
    \item 所有的$v_\theta$、$D_\theta$ 都连续地依赖于上下文$c$。
    \item 这意味着,约束函数是机制$M$和上下文$c$的一个联合连续函数。
\end{itemize}

考虑我们找到的机制$M_{\epsilon}$。由于它的所有约束都是严格成立的,根据连续性,当上下文从$c$稍微扰动到$c_n$时(当$n$足够大时),这些约束仍然会成立。这意味着,对于所有足够大的$n$,机制$M_{\epsilon}$在上下文$c_n$下也是一个可行的机制。

形式化地,因为约束函数$g(c,M)$是连续的,且$g(c, M_{\epsilon})>0$,那么存在一个$c$的邻域$N(c)$,使得对于所有$c' \in N(c)$,都有$g(c',M_{\epsilon})>0$。因为$c_n\to c$,所以当$n$足够大时,$c_n \in N(c)$。

因此,对于所有足够大的$n$,我们有:
$$R(c_n)=\sup_{M\in F(c_n)} \text{Revenue}(c_n,M)\geq \text{Revenue}(c_n,M_{\epsilon})$$
$R(c_n)$是在$c_n$下的最优收益,它必然不低于任何一个可行机制(如$M_{\epsilon}$)的收益。

现在我们取下极限:
$$\lim\inf_{n\to\infty} R(c_n)\geq \text{Revenue}(c,M_{\epsilon})$$
根据$M_{\epsilon}$的构造,我们有$\text{Revenue}(c,M_{\epsilon}) = R(M_{\epsilon})\geq (1 - \epsilon)R(c)$。所以:
$$\lim\inf_{n\to\infty} R(c_n)\geq (1-\epsilon)R(c)$$
这个不等式对任意的$\epsilon>0$都成立。因此,我们可以令$\epsilon\to 0$,得到最终的结论:
$$\lim\inf_{n\to\infty}R(c_n)\geq R(c)$$
这正是下半连续性的定义。

证毕!