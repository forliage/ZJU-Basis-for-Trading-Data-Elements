\section{通用交互协议与显示原理 (General Protocols and the Revelation Principle)}

\subsection{描述一切可能:通用交互协议 (Generic Interactive Protocol)}

为了不遗漏任何可能的销售策略,我们需要一个能囊括所有互动方式的通用模型。这个模型就是通用交互协议,它可以被想象成一个多回合的、动态的博弈树。

\textbf{定义3.1(通用交互协议)}:一个通用交互协议是一个有限的决策树,由一系列节点 (Node) 和连接节点的边 (Edge) 构成。树中的非叶子节点分为三类:
\begin{enumerate}
    \item 卖方节点 (Seller Node):轮到卖方行动。
       \begin{itemize}
         \item 在卖方节点$n$,卖方会根据她的私有信息$\omega$,以一定的概率选择一条边(即发送一个信号)通往下一个节点。
         \item 形式化地,对于节点$n$的每一个子节点$c$,卖方都预先设定了一个概率$p_n(\omega, c)$,表示当她自己的信号是$\omega$时,她选择走向子节点$c$的概率。
         \item 对于任意$\omega$,必须有$\sum\limits_{c}p_n(\omega,c)=1$。
       \end{itemize}
    \item 买方节点 (Buyer Node):轮到买方行动。
       \begin{itemize}
         \item 在买方节点$n$,买方会根据他的私有类型$\theta$(以及在预算约束模型中的预算$b$)来决定走向哪个子节点。
         \item 买方的策略$\phi$是一个函数,它将买方的类型$(\theta,b)$和当前节点$n$映射到一个关于子节点的概率分布。
       \end{itemize}
    \item 支付节点 (Transfer Node):发生资金转移。
       \begin{itemize}
         \item 支付节点$n$只有一个子节点。
         \item 与每个支付节点$n$相关联的是一个固定的金额$t(n)$。
         \item 当协议路径经过此节点时,一笔金额为$t(n)$的支付从买方转移给卖方。
         \item 如果$t(n)$是负数,则表示卖方向买方支付了$|t(n)|$。
       \end{itemize}
\end{enumerate}

\textbf{协议的执行过程:}协议从根节点开始。根据当前节点的类型(卖方、买方或支付),某一方行动或发生支付,协议状态转移到下一个节点。这个过程持续进行,直到到达一个叶子节点 (Leaf Node)。

在到达叶子节点1时,整个交互结束。买方根据从根节点到叶子节点1的整条路径上观察到的所有卖方行动(即收到的所有信号),形成对$\omega$的最终后验信念。然后,他基于这个后验信念选择一个最优的行动$a$来最大化自己的期望效用。

\subsection{机制设计的基石:显示原理 (The Revelation Principle)}

显示原理是机制设计领域最核心的定理之一。它的思想是,任何复杂的、间接的机制,其最终达成的效果,都可以通过一个简单得多的直接显示机制 (Direct Revelation Mechanism) 来实现,并且收益不会降低。

\textbf{定义 3.2 (直接显示机制)}:一个直接显示机制是一种特殊的协议,其结构非常简单:
\begin{enumerate}
    \item 协议的根节点是一个买方节点。
    \item 在这个节点,买方被要求直接报告他的私有类型$\theta$(以及预算$b$)。
    \item 树中没有其他的买方节点。买方只行动一次,就是在一开始“自报家门”。
    \item 机制(现在由卖方扮演)根据买方报告的类型$(\hat{\theta},\hat{b})$,以及自己掌握的$\omega$,来执行后续的支付和信息披露。
\end{enumerate}

\textbf{定理 3.1 (显示原理,信息销售版)}:假设存在一个任意复杂的通用交互协议$\Pi$,以及在该协议下一个理性的买方最优策略$\phi$,这个组合$(\Pi,\phi)$能为卖方带来期望收益$R$。
那么,必然存在一个直接显示机制$\Pi_d$,在该机制下,买方如实报告自己的真实类型$(\theta,b)$是其最优策略(我们称之为激励相容 Incentive Compatible, IC),并且这个直接机制能为卖方带来至少为$R$的期望收益。

\textbf{显示原理的直觉证明 (Proof by Simulation)}。这个定理的证明非常有启发性,其核心思想是模拟。

假设我们有一个复杂的、多回合的旧机制$\Pi$。我们可以构造一个新的直接机制$\Pi_d$如下:

\begin{enumerate}
    \item 新机制$\Pi_d$的第一步,是请买方报告他的类型$\theta$。
    \item 一旦买方报告了$\hat{\theta}$,新机制(卖方)就在自己的“后台”开始模拟旧机制$\Pi$的整个过程。当在旧机制中轮到买方行动时,新机制就查阅买方在旧机制下的最优策略$\phi(\hat{\theta})$,然后代替买方做出那个选择。当轮到卖方行动时,新机制就按照旧机制的规则行动。
    \item 新机制在后台完整地模拟完一遍旧机制,得到一个最终的 outcome(包括给买方的信息和买方需要支付的总金额)。然后,新机制直接把这个最终 outcome 交给买方。
\end{enumerate}

为什么买方愿意如实报告?

因为如果他撒谎,比如他的真实类型是$\theta$,却报告了$\hat{\theta}$。那么新机制就会按照$\hat{\theta}$的最优策略$\phi(\hat{\theta})$去模拟。这等价于,在旧机制$\Pi$中,一个类型为$\theta$的买方,假装自己是$\hat{\theta}$来玩游戏。但我们已经假设了$\phi(\theta)$是类型为$\theta$的买方在旧机制中的最优策略,所以假装成别人($\hat{\theta}$)来玩,对他来说收益不会更高。因此,在新机制中,他没有动力撒谎,如实报告是他的最优选择。

显示原理极大地简化了我们的问题。我们不再需要在所有可能的、奇形怪状的协议树中进行搜索,而只需要在一类结构非常简单的直接机制中寻找最优者。在直接机制中,我们只需要决定两件事:
\begin{enumerate}
    \item 分配规则:对于每一种可能的报告$(\hat{\theta},\hat{b})$和真实的$\omega$,卖方应该向买方披露什么样的信息?
    \item 
    支付规则:对于每一种可能的报告$(\hat{\theta},\hat{b})$和真实的$\omega$,卖方应该向买方收取多少钱?
\end{enumerate}

\subsection{更进一步:单轮显示机制 (One-Round Revelation)}

在许多经典的机制设计问题(如商品拍卖)中,显示原理还能被加强。不仅买方只需要报告一次,整个支付和分配过程也可以在“一轮”内完成。然而,在信息销售中,情况变得微妙起来。

\textbf{定义 3.3 (单轮显示机制)}:一个单轮显示机制是一个深度最多为3的直接显示机制。其路径通常是:
$$\text{买方报告类型}\to\text{支付发生}\to\text{卖方披露信息}$$
或者
$$\text{买方报告类型}\to\text{卖方披露信息}\to\text{支付发生}$$

核心问题:对于信息销售问题,最优机制是否总能简化为单-轮的显示机制?答案是:不一定。这取决于买方的承诺和信号的相关性。

\textbf{定理 3.2 (单轮机制的充分条件)}:在以下两种情况中,任何机制(无论多复杂)的收益,都可以被一个单轮显示机制所实现:
\begin{enumerate}
    \item 当买方是承诺的 (Committed Buyer)。
    \item 当买方不承诺,但卖方信号$\omega$和买方类型$\theta$是相互独立的。
\end{enumerate}

证明思路 (Case 1: Committed Buyer)当买方是承诺的时候,他不能中途退出。我们可以构造一个单轮机制如下:
\begin{enumerate}
    \item 买方报告类型$\theta$。
    \item 机制(卖方)根据$\theta$和自己的$\omega$,在后台模拟完整个原始协议。
    \item 在模拟过程中,机制会记录下所有需要发生的支付,并计算出总支付额 t\_total。同时,它也会计算出最终应该披露给买方的信息 s\_final。
    \item 模拟结束后,机制直接向买方收取 t\_total,并同时把 s\_final 给他。由于买方是承诺的,他必须接受这个最终结果。这个过程等价于原始协议,并且显然是单轮的。
\end{enumerate}

证明思路 (Case 2: Uncommitted but Independent Signals)
当信号独立时$\mu(\omega,\theta)=\mu(\omega)\mu(\theta)$, 买方对$\omega$的先验信念$\mu(\omega)$与其自身类型$\theta$无关。这意味着,无论买方报告自己是$\theta$还是$\hat{\theta}$,机制在计算期望支付时,都是对同一个$\omega$的分布$\mu(\omega)$求期望。
我们可以构造一个单轮机制:
\begin{enumerate}
    \item 买方报告类型$\theta$。
    \item 机制计算出,如果买方类型真的是$\theta$,那么在原始协议中他的期望支付额是多少,记为$\mathbb{E}[t|\theta]$。
    \item 机制先向买方收取$\mathbb{E}[t|\theta]$。
    \item 然后,机制再根据$\omega$和买方报告的$\theta$,模拟原始协议中的信息披露部分,并将最终信息给买方。
\end{enumerate}

这个机制是可行的,因为支付发生在信息披露之前,不承诺的买方没有机会“白看”信息。而由于信号独立,一个类型为$\theta$的买方谎报成$\hat{\theta}$,并不会改变他对支付额的期望(因为都是对$\mu(\omega)$求期望),所以他欺骗的动机只剩下获取信息上的好处,而这在原始协议中已经被保证为不划算的。

当单轮显示原理失效时.关键洞察:当买方不承诺且信号相关时,单轮显示原理失效!原因:在这种情况下,上面两种构造方法都行不通了。
\begin{itemize}
    \item 先披露信息,后收费:行不通。不承诺的买方拿到信息后会直接跑路。
    \item 先收费,后披露信息:也行不通。收费多少呢?我们不能像独立信号时那样收取期望支付。因为信号是相关的,一个真实类型为$\theta$的买方,如果他谎报成$\hat{\theta}$,他对$\omega$的信念是$\mu(\omega|\theta)$,而机制是按$\mu(\omega|\hat{\theta})$来计算期望支付的。这两个期望值不同,导致买方可以通过谎报来操纵他需要预付的费用。
\end{itemize}

结论:在这种最复杂的情况下,为了实现最优收益,卖方可能必须采取一种“挤牙膏”式的策略:披露一点信息$\to$收一点钱$\to$再披露一点信息$\to$再收一点钱…… 这种交错进行的多轮互动,是无法被一个单-轮机制所模拟的。

\subsection{预算约束下的简化}

引入预算约束,这个看似简单的现实因素,却极大地简化了最优机制的结构,使得多轮“挤牙膏”式的复杂机制变得不再必要。

其核心思想在于,预算约束限制了支付的可能性。特别是,论文中设计了一种巧妙的支付方式,称为“存款-返还” (Deposit-Return),它有效地解决了不承诺买方的跑路问题。

咨询机制 (Consulting Mechanism)是一种简洁而强大的机制,是预算约束下信息销售问题的最优解。其一般形式如下:
\begin{enumerate}
    \item 咨询开始:买方向卖方(顾问)报告自己的私有类型$\theta$。
    \item (可选) 存款:在需要时,买方需要先存入一笔押金(通常是其预算上限)。
    \item 内部处理:卖方根据买方的报告$\theta$和自己的$\omega$,决定两件事:
       \begin{itemize}
         \item 一个推荐给买方的最佳行动$a^\ast$。
         \item 一个支付(或返还)金额。
       \end{itemize}
    \item 给出建议并结算:卖方将推荐行动$a^\ast$告诉买方,并完成最终的资金结算。
\end{enumerate}

这个机制的巧妙之处在于它的支付规则。它绕过了直接对“信息”定价的难题,转而对“行动建议”收费。并且通过“存款-返还”的结构,即便支付金额依赖于卖方的信息$\omega$,也能保证不承诺的买方不会跑路。

我们将在后续章节详细剖析这种咨询机制是如何通过线性规划来求解的,以及它为何在有预算约束时能达到最优。