\section{问题的形式化 (Formalizing the Problem)}

为了能够严谨地分析信息销售问题,我们首先需要建立一个统一的数学语言。

\subsection{核心要素:上下文 (Context)$(u,\mu)$}

整个信息销售问题可以被一个我们称为上下文(Context)的元组$(u, \mu)$所概括。这个上下文是买卖双方的共同知识(Common Knowledge)

它包含以下核心要素:

\textbf{世界状态 (State of the World)}:$\omega \in \Omega$。这是一个随机变量,代表了客观世界中一个不确定的、但与决策收益相关的状态。
$\Omega$是所有可能状态的有限集合。

\textbf{卖方私有信号 (Seller's Private Signal)}:在本模型的基本设定中,我们假设卖方完全知晓世界状态$\omega$。所以,$\omega$就是卖方的私有信息。

\textbf{买方私有类型 (Buyer's Private Type)}:$\theta \in \Theta$。这也是一个随机变量,代表了买方在交易开始前就拥有的私有信息。

\subsection{买方的目标:最大化效用}

\subsection{卖方的目标:最大化收益}

\subsection{初步尝试:“密封信封”机制 (The "Sealed Envelope" Mechanism)}

\subsection{示例}