\section{案例分析与代码实现 (Case Studies and Implementation)}

本章我们将通过论文中给出的具体数值案例,并辅以Python代码,来清晰地展示前几章理论的应用。

\subsection{案例一:私有预算下的“存款-返还”机制}

场景设定:
\begin{itemize}
    \item 世界状态与行动:$\Omega=A=\{0,1\}$(一个“金库”问题,两把钥匙0和1,只有一个能打开)。
    \item 买方类型与效用:$\Theta=\{0,1\}$。买方$\Theta$的效用函数为$u(\theta,\omega,a)=z_{\theta}* 1\{\omega=a\}$,其中:
       \begin{itemize}
          \item $z_0 = 120$(类型0买方打开金库的价值是120)
          \item $z_1= 80$(类型1买方打开金库的价值是80)
       \end{itemize}
    \item 私有预算:
       \begin{itemize}
          \item 类型0买方的预算是 $b_0 = 50$。
          \item 类型1买方的预算是 $b_1 = 100$。
       \end{itemize}
    \item 概率分布:$\omega$和$\theta$相互独立且均匀分布,$\mu(\omega,\theta)= 1/4$
\end{itemize}

一个类型为$\theta$的买方,在没有额外信息时,会随机猜一把钥匙,期望收益是$0.5* z_\theta$。在获得完整信息$\omega$后,他总能选对钥匙,收益是$z_\theta$。因此,完整信息的价值是:
$$\xi(0) = z_0 - 0.5 * z_0 = 0.5 * 120 = 60$$
$$\xi(1) = z_1 - 0.5 * z_1 = 0.5 * 80 = 40$$

假设我们使用一个直接支付的咨询机制,即提供一个价格菜单$\{t_0,t_1\}$。

卖方的困境:卖方想对类型0的买方(高价值,$\xi(0)=60$)收取更高的价格,对类型1的买方(低价值,$\xi(1)=40$)收取较低的价格。

买方的激励:然而,类型0的买方可以谎报自己是类型1,从而享受为类型1设计的低价。

IC约束的威力:为了防止类型0的买方谎报,价格$t_0$不能比$t_1$高太多。具体的IC约束是:

$$\xi(0) - t_0 \geq \xi(0) - t_1 \rightrightarrows t_1\geq t_0$$
(这显然不可能,除非价格一样)实际上,当信息披露策略相同时(都完全披露$\omega$),类型$\theta$谎报成$\theta^\prime$的净效用是
$$\xi(\theta)-t_{\theta^\prime}$$

\begin{itemize}
    \item 类型0不谎报的IC约束: $60 - t_0 \geq 60 - t_1 \rightrightarrows t_1 \geq t_0$
    \item 类型1不谎报的IC约束: $40 - t_1 \geq 40 - t_0 \rightrightarrows t_0 \geq t_1$
    \item 这两个约束合在一起,意味着 $t_0 = t_1$。也就是说,卖方只能对两种类型收取相同的价格。
\end{itemize}

预算约束:类型0的买方预算只有50。所以,这个共同的价格$t$不能超过50。

最优直接支付策略:
\begin{itemize}
    \item 如果定价 $t > 40$,类型1买方不会购买(因为 $t > \xi(1)$)。只有类型0买方会买。最大化收益是定价 $t=50$(受限于类型0的预算),收益为 $0.5 * 50 = 25$。
    \item 如果定价 $t \leq 40$,两种类型都会购买。最大化收益是定价 $t=40$,收益为 $1.0 * 40 = 40$。
\end{itemize}

结论: 最优的直接支付机制,其最大收益是 40。

现在,我们利用“存款-返还”的结构来打破僵局。卖方可以提供如下两个合约选项:

合约A: 报告你是类型0,预算50。请先存入50。然后我们会告诉你完整信息$\omega$,并且不返还任何金额。(净支付 $t_0 = 50$)
合约B: 报告你是类型1,预算100。请先存入100。然后我们会告诉你完整信息$\omega$,并返还61。(净支付$t_1 = 100 - 61 = 39$)

分析买方的选择:

类型0买方 (预算50):
\begin{itemize}
    \item 他只能选择合约A,因为他没有100元去存入来参与合约B。
    \item 他会计算:花50元,得到价值60元的信息,净收益 60 - 50 = 10 > 0。所以他愿意参与。
\end{itemize}

类型1买方 (预算100):
\begin{itemize}
    \item 他可以选合约A或B。
    \item 如果选A(谎报自己是类型0):他需要存50,净支付50。他得到价值40的信息,净收益 40 - 50 = -10。不划算。
    \item 如果选B(如实报告):他需要存100,净支付39。他得到价值40的信息,净收益 40 - 39 = 1 > 0。划算。
    \item 因此,他会选择合约B。
\end{itemize}

新机制的收益:
\begin{itemize}
    \item 一半的概率($\mu(0)=0.5$)买方是类型0,选择合约A,卖方收入50。
    \item 一半的概率($\mu(1)=0.5$)买方是类型1,选择合约B,卖方收入39。
    \item 总期望收益 = 0.5 * 50 + 0.5 * 39 = 25 + 19.5 = 44.5。
\end{itemize}

结论: 这个“存款-返还”机制的收益 44.5,严格高于最优直接支付机制的收益 40。这清晰地表明,利用预算作为筛选工具(通过存款要求)可以有效地进行价格歧视,从而提取更多收益。

\subsection{代码实现:用线性规划求解咨询机制}