\section{案例分析与代码实现 (Case Studies and Implementation)}

本章我们将通过论文中给出的具体数值案例,并辅以Python代码,来清晰地展示前几章理论的应用。

\subsection{案例一:私有预算下的"存款-返还"机制}

\textbf{场景设定:}
\begin{itemize}
    \item \textbf{世界状态与行动}:$\Omega=A=\{0,1\}$(一个"金库"问题,两把钥匙0和1,只有一个能打开)。
    \item \textbf{买方类型与效用}:$\Theta=\{0,1\}$。买方$\theta$的效用函数为$u(\theta,\omega,a)=z_{\theta}\cdot \mathbf{1}\{\omega=a\}$,其中:
       \begin{itemize}
          \item $z_0 = 120$(类型0买方打开金库的价值是120)
          \item $z_1= 80$(类型1买方打开金库的价值是80)
       \end{itemize}
    \item \textbf{私有预算}:
       \begin{itemize}
          \item 类型0买方的预算是$b_0 = 50$。
          \item 类型1买方的预算是$b_1 = 100$。
       \end{itemize}
    \item \textbf{概率分布}:$\omega$和$\theta$相互独立且均匀分布,$\mu(\omega,\theta)= 1/4$
\end{itemize}

一个类型为$\theta$的买方,在没有额外信息时,会随机猜一把钥匙,期望收益是$0.5\cdot z_\theta$。在获得完整信息$\omega$后,他总能选对钥匙,收益是$z_\theta$。因此,完整信息的价值是:
$$\xi(0) = z_0 - 0.5 \cdot z_0 = 0.5 \cdot 120 = 60$$
$$\xi(1) = z_1 - 0.5 \cdot z_1 = 0.5 \cdot 80 = 40$$

假设我们使用一个直接支付的咨询机制,即提供一个价格菜单$\{t_0,t_1\}$。

\textbf{卖方的困境:}卖方想对类型0的买方(高价值,$\xi(0)=60$)收取更高的价格,对类型1的买方(低价值,$\xi(1)=40$)收取较低的价格。

\textbf{买方的激励:}然而,类型0的买方可以谎报自己是类型1,从而享受为类型1设计的低价。

\textbf{IC约束的威力:}为了防止类型0的买方谎报,价格$t_0$不能比$t_1$高太多。具体的IC约束是:

实际上,当信息披露策略相同时(都完全披露$\omega$),类型$\theta$谎报成$\theta'$的净效用是$\xi(\theta)-t_{\theta'}$。

\begin{itemize}
    \item \textbf{类型0不谎报的IC约束}:$60 - t_0 \geq 60 - t_1 \Rightarrow t_1 \geq t_0$
    \item \textbf{类型1不谎报的IC约束}:$40 - t_1 \geq 40 - t_0 \Rightarrow t_0 \geq t_1$
    \item 这两个约束合在一起,意味着$t_0 = t_1$。也就是说,卖方只能对两种类型收取相同的价格。
\end{itemize}

\textbf{预算约束:}类型0的买方预算只有50。所以,这个共同的价格$t$不能超过50。

\textbf{最优直接支付策略:}
\begin{itemize}
    \item 如果定价$t > 40$,类型1买方不会购买(因为$t > \xi(1)$)。只有类型0买方会买。最大化收益是定价$t=50$(受限于类型0的预算),收益为$0.5 \times 50 = 25$。
    \item 如果定价$t \leq 40$,两种类型都会购买。最大化收益是定价$t=40$,收益为$1.0 \times 40 = 40$。
\end{itemize}

\textbf{结论:}最优的直接支付机制,其最大收益是40。

现在,我们利用"存款-返还"的结构来打破僵局。卖方可以提供如下两个合约选项:

\textbf{合约A}:报告你是类型0,预算50。请先存入50。然后我们会告诉你完整信息$\omega$,并且不返还任何金额。(净支付$t_0 = 50$)

\textbf{合约B}:报告你是类型1,预算100。请先存入100。然后我们会告诉你完整信息$\omega$,并返还61。(净支付$t_1 = 100 - 61 = 39$)

\textbf{分析买方的选择:}

类型0买方(预算50):
\begin{itemize}
    \item 他只能选择合约A,因为他没有100元去存入来参与合约B。
    \item 他会计算:花50元,得到价值60元的信息,净收益$60 - 50 = 10 > 0$。所以他愿意参与。
\end{itemize}

类型1买方(预算100):
\begin{itemize}
    \item 他可以选合约A或B。
    \item 如果选A(谎报自己是类型0):他需要存50,净支付50。他得到价值40的信息,净收益$40 - 50 = -10$。不划算。
    \item 如果选B(如实报告):他需要存100,净支付39。他得到价值40的信息,净收益$40 - 39 = 1 > 0$。划算。
    \item 因此,他会选择合约B。
\end{itemize}

\textbf{新机制的收益:}
\begin{itemize}
    \item 一半的概率($\mu(0)=0.5$)买方是类型0,选择合约A,卖方收入50。
    \item 一半的概率($\mu(1)=0.5$)买方是类型1,选择合约B,卖方收入39。
    \item 总期望收益$= 0.5 \times 50 + 0.5 \times 39 = 25 + 19.5 = 44.5$。
\end{itemize}

\textbf{结论:}这个"存款-返还"机制的收益44.5,严格高于最优直接支付机制的收益40。这清晰地表明,利用预算作为筛选工具(通过存款要求)可以有效地进行价格歧视,从而提取更多收益。

\subsection{代码实现:用线性规划求解咨询机制}

\textbf{场景设定:}
\begin{itemize}
    \item $\Omega=A=\Theta=\{0,1\}$
    \item $\mu(\omega,\theta)$:$\mu(0,0)=0.4$,$\mu(0,1)=0.1$,$\mu(1,0)=0.1$,$\mu(1,1)=0.4$。$\omega$和$\theta$相关。
    \item $u(\omega,a)=10$ if $\omega=a$ else $0$。
    \item \textbf{公开预算}:假设所有买家预算$b = 10$。
    \item \textbf{卖方预算}:$M = 10$
    \item \textbf{信号相关}:所以我们需要用最通用的CM-probR机制的LP形式来求解。
\end{itemize}

具体的代码参考附件中的program2.py

\begin{lstlisting}[language=Python,style=pythonstyle]
import cvxpy as cp
import numpy as np

class CorrelatedContext:
    def __init__(self):
        self.Omega = [0, 1]
        self.Theta = [0, 1]
        self.A = [0, 1]
        self.b = 10.0  
        self.M = 10.0  

        # 联合概率分布
        self.mu_joint = np.array([[0.4, 0.1],  # theta=0: mu(t=0,w=0), mu(t=0,w=1)
                                  [0.1, 0.4]]) # theta=1: mu(t=1,w=0), mu(t=1,w=1)
        self.mu_theta = np.sum(self.mu_joint, axis=1) 
        self.mu_omega = np.sum(self.mu_joint, axis=0) 

        # 条件概率:P(w|t)
        self.mu_cond_w_given_t = self.mu_joint / self.mu_theta[:, np.newaxis]

    def u(self, omega, action):
        """效用函数:匹配时获得10,否则为0"""
        return 10.0 if omega == action else 0.0

def solve_cm_probr_corrected(ctx):
    """
    求解概率性返还的咨询机制(CM-probR)
    z[t,pay,a,w] = 报告类型t时,在状态w下推荐行动a并要求支付pay的概率*联合概率
    """
    z = cp.Variable((len(ctx.Theta), 2, len(ctx.A), len(ctx.Omega)), nonneg=True)
    
    # 支付结构:0=收费b, 1=返还M
    payments = np.array([ctx.b, -ctx.M])
    revenue = cp.sum(cp.multiply(z, payments[np.newaxis, :, np.newaxis, np.newaxis]))
    objective = cp.Maximize(revenue)
    
    constraints = []

    # 可行性约束:后验期望等于先验
    for t_idx in range(len(ctx.Theta)):
        for w_idx in range(len(ctx.Omega)):
            constraints.append(cp.sum(z[t_idx, :, :, w_idx]) == ctx.mu_joint[t_idx, w_idx])

    # 听话约束:推荐的行动必须是最优的
    for t_rep_idx in range(len(ctx.Theta)):
        for a_rec_idx, a_rec in enumerate(ctx.A):
            # 收费b时的听话约束
            util_obedient_b = sum(z[t_rep_idx, 0, a_rec_idx, w_idx] * ctx.u(ctx.Omega[w_idx], a_rec) 
                                 for w_idx in range(len(ctx.Omega)))
            # 返还M时的听话约束
            util_obedient_m = sum(z[t_rep_idx, 1, a_rec_idx, w_idx] * ctx.u(ctx.Omega[w_idx], a_rec) 
                                 for w_idx in range(len(ctx.Omega)))
            
            for a_dev_idx, a_dev in enumerate(ctx.A):
                if a_rec_idx != a_dev_idx:
                    util_deviate_b = sum(z[t_rep_idx, 0, a_rec_idx, w_idx] * ctx.u(ctx.Omega[w_idx], a_dev) 
                                        for w_idx in range(len(ctx.Omega)))
                    constraints.append(util_obedient_b >= util_deviate_b)
                    
                    util_deviate_m = sum(z[t_rep_idx, 1, a_rec_idx, w_idx] * ctx.u(ctx.Omega[w_idx], a_dev) 
                                        for w_idx in range(len(ctx.Omega)))
                    constraints.append(util_obedient_m >= util_deviate_m)

    # 计算净效用矩阵
    U_net = [[None for _ in ctx.Theta] for _ in ctx.Theta]

    for t_true_idx in range(len(ctx.Theta)):
        for t_rep_idx in range(len(ctx.Theta)):
            # 收费b时的期望效用
            expected_gross_util_b = sum(z[t_rep_idx, 0, a_idx, w_idx] * ctx.u(ctx.Omega[w_idx], ctx.A[a_idx])
                                       for a_idx in range(len(ctx.A))
                                       for w_idx in range(len(ctx.Omega))
                                       if ctx.mu_joint[t_true_idx, w_idx] > 0) 
            
            expected_payment_b = sum(z[t_rep_idx, 0, a_idx, w_idx] * ctx.b
                                    for a_idx in range(len(ctx.A))
                                    for w_idx in range(len(ctx.Omega))
                                    if ctx.mu_joint[t_true_idx, w_idx] > 0)

            # 返还M时的期望效用
            expected_gross_util_m = sum(z[t_rep_idx, 1, a_idx, w_idx] * ctx.u(ctx.Omega[w_idx], ctx.A[a_idx])
                                       for a_idx in range(len(ctx.A))
                                       for w_idx in range(len(ctx.Omega))
                                       if ctx.mu_joint[t_true_idx, w_idx] > 0)
            
            expected_payment_m = sum(z[t_rep_idx, 1, a_idx, w_idx] * (-ctx.M)
                                    for a_idx in range(len(ctx.A))
                                    for w_idx in range(len(ctx.Omega))
                                    if ctx.mu_joint[t_true_idx, w_idx] > 0)

            # 归一化为净效用
            mu_t_true = ctx.mu_theta[t_true_idx]
            if mu_t_true > 1e-9:
                U_net[t_true_idx][t_rep_idx] = ((expected_gross_util_b + expected_gross_util_m - 
                                               expected_payment_b - expected_payment_m) / mu_t_true)
            else:
                U_net[t_true_idx][t_rep_idx] = 0

    # IC和IR约束
    for t_true_idx in range(len(ctx.Theta)):
        # 计算保留效用(基于先验的最优决策)
        u_prior = -np.inf
        prior_dist = ctx.mu_cond_w_given_t[t_true_idx, :]
        for a_idx, a in enumerate(ctx.A):
            util = sum(prior_dist[w_idx] * ctx.u(ctx.Omega[w_idx], a) for w_idx in range(len(ctx.Omega)))
            u_prior = max(u_prior, util)
        
        # IR约束:参与不劣于不参与
        constraints.append(U_net[t_true_idx][t_true_idx] >= u_prior)

        # IC约束:如实报告不劣于谎报
        for t_rep_idx in range(len(ctx.Theta)):
            if t_true_idx != t_rep_idx:
                constraints.append(U_net[t_true_idx][t_true_idx] >= U_net[t_true_idx][t_rep_idx])

    # 求解线性规划
    problem = cp.Problem(objective, constraints)
    problem.solve(solver=cp.SCS, verbose=False)

    print("--- LP Solver Results (Corrected) ---")
    print(f"Solver status: {problem.status}")
    if problem.value is not None:
        print(f"Optimal Revenue: {problem.value:.4f}")
        
    if problem.status in [cp.OPTIMAL, cp.OPTIMAL_INACCURATE]:
        z_val = z.value
        print("\n--- Optimal Mechanism Policy ---")
        for t_rep_idx in range(len(ctx.Theta)):
            print(f"\nIf buyer reports type theta={t_rep_idx}:")
            p = np.zeros_like(z_val[t_rep_idx])
            for w_idx in range(len(ctx.Omega)):
                mu_tw = ctx.mu_joint[t_rep_idx, w_idx]
                if mu_tw > 1e-9:
                    p[:, :, w_idx] = z_val[t_rep_idx, :, :, w_idx] / mu_tw
                else: 
                    p[:, :, w_idx] = 0

            for w_idx, w in enumerate(ctx.Omega):
                print(f"  If seller observes omega={w}:")
                has_action = False
                for a_idx, a in enumerate(ctx.A):
                    prob_pay_b = p[0, a_idx, w_idx]
                    prob_pay_m = p[1, a_idx, w_idx]
                    if prob_pay_b > 1e-6:
                        print(f"    -> Recommend action '{a}', charge {ctx.b} (Prob: {prob_pay_b:.2f})")
                        has_action = True
                    if prob_pay_m > 1e-6:
                        print(f"    -> Recommend action '{a}', pay back {ctx.M} (Prob: {prob_pay_m:.2f})")
                        has_action = True
                if not has_action:
                     print("    -> No action recommended.")

if __name__ == "__main__":
    ctx_corr = CorrelatedContext()
    solve_cm_probr_corrected(ctx_corr)    
\end{lstlisting}

这个代码框架清晰地展示了如何将一个复杂的机制设计问题,通过理论的指导(CM-probR是最优的),最终转化为一个可以被标准工具解决的具体、结构化的线性规划问题。这是理论与实践结合的完美体现。