\section{求解最优机制:从独立信号到咨询机制}

本章的目标是,将机制设计问题从一个抽象的"寻找最佳规则"问题,转变为一个具体的、可以求解的线性规划 (Linear Programming, LP) 问题。

\subsection{独立信号下的最优机制}

我们从最简单但富有启发性的情况开始:

\begin{itemize}
    \item 卖方信号$\omega$和买方类型$\theta$互相独立,即$\mu(\omega,\theta)=\mu(\omega)\mu(\theta)$。
    \item 没有预算约束。
    \item 买方是不承诺的。
\end{itemize}

根据第三章的定理3.2,我们知道在这种情况下,最优机制一定是单轮的。更具体地说,它是一种"先收费,后披露"的机制,被称为定价映射机制 (Pricing Mappings Mechanism),这等价于向买方提供一个"合约菜单"。

\subsubsection{合约菜单的视角}

一个直接显示机制可以被看作是卖方向买方提供一个合约菜单 (Menu of Contracts) $\{(Y_\theta, t_\theta)\}_{\theta\in\Theta}$。

\begin{itemize}
    \item 菜单上的每一项$(Y_\theta,t_\theta)$都是一个"合约",它名义上是为类型为$\theta$的买方准备的。
    \item $t_\theta$是购买该合约的价格。
    \item $Y_\theta$是一个信号生成器。它是一个随机变量,其分布依赖于卖方的真实信息$\omega$。买方购买合约后,卖方会根据$\omega$从$Y_\theta$中抽取一个信号$s$并发送给买方。
\end{itemize}

\textbf{买方的决策过程:}

\begin{enumerate}
    \item 一个真实类型为$\theta$的买方,会审视菜单上的所有合约。
    \item 对于每个合约$(Y_{\theta'},t_{\theta'})$,他会计算如果自己购买该合约能获得的期望效用。
    \item 他会选择那个能给他带来最高净效用的合约。
    \item 他还会在"购买最优合约"和"什么都不买"(获得保留效用)之间进行比较。
\end{enumerate}

\textbf{卖方的设计问题:}
卖方的目标是设计这个菜单$\{(Y_\theta,t_\theta)\}$,使得:

\begin{enumerate}
    \item \textbf{激励相容} (Incentive Compatibility, IC):对于每个$\theta$,类型为$\theta$的买方确实觉得合约$(Y_\theta,t_\theta)$是菜单上最好的选择(或者之一)。
    \item \textbf{个体理性} (Individual Rationality, IR):对于每个$\theta$,类型为$\theta$的买方购买合约$(Y_\theta,t_\theta)$的净效用,不低于他不参与交易的保留效用。
    \item 在满足以上两个条件下,最大化总期望收益$\sum\limits_{\theta}\mu(\theta)t_\theta$。
\end{enumerate}

\subsubsection{信息的数学表示:从信号到后验}

"信号生成器$Y_\theta$"这个概念仍然有些抽象。为了进行数学优化,我们需要一种更具体的方式来描述信息。一个关键的转化是:任何关于信息披露的机制,其本质都是在改变买方的后验信念。

\begin{itemize}
    \item 一个信号$s$的全部意义,在于它如何更新买方对$\omega$的信念。
    \item 在收到信号$s$之前,买方的信念是先验$p(\omega)$(因为信号独立,先验就是$\mu(\omega)$)。
    \item 在收到信号$s$之后,根据贝叶斯法则,买方的信念更新为一个后验分布$q(\omega)=\Pr(\omega|s)$。
\end{itemize}

因此,我们可以用后验分布的分布来等价地描述一个信号生成器$Y_\theta$:$Y_\theta$不再是生成信号$s$,而是以一定的概率$x_\theta(q)$,直接生成一个后验分布$q$。买方支付$t_\theta$后,他会得到一个后验分布$q$,然后基于这个$q$去做决策。

如果买方的后验信念是$q$,他的最优期望效用是:
$$v_\theta(q)=\max_{a\in A}\mathbb{E}_{\omega\sim q}[u(\theta,\omega,a)]=\max_{a\in A}\sum\limits_{\omega\in\Omega}q(\omega)u(\theta,\omega,a)$$

函数$v_\theta(q)$是$q$的一个分片线性凸函数(它是多个线性函数的最大值)。

一个后验的分布$\{x_\theta(q)\}$并非任意的,它必须满足一个物理约束:所有可能的后验的期望,必须等于先验。这被称为\textbf{贝叶斯可信性} (Bayesian Plausibility) 或\textbf{可行性约束} (Feasibility Constraint)。
$$\mathbb{E}_{q\sim x_\theta}[q]=\sum\limits_{q}x_\theta(q)\cdot q = p \quad \text{(其中}p(\omega)=\mu(\omega)\text{)}$$
这个约束保证了卖方不能"无中生有"地创造信息。

\subsubsection{构建线性规划 (LP)}

现在,我们可以将所有元素组合起来,构建一个用于求解最优合约菜单的线性规划。

\textbf{决策变量:}
\begin{itemize}
    \item $t_\theta$:为每个买方类型$\theta$设计的价格。
    \item $x_\theta(q)$:对于每个类型$\theta$和每个可能的后验$q$,选择生成后验$q$的概率。
\end{itemize}

我们不需要考虑所有可能的后验。因为效用函数$v_\theta(q)$是分片线性的,其"拐点"只发生在有限的位置。我们可以预先计算出一个有限的、足够大的"有趣后验"集合$Q^*$,最优机制只会从这个集合中选择后验来生成。$Q^*$的大小是关于$|A|,|\Omega|,|\Theta|$的多项式。这个结论使得变量的数量从无限变成了有限(尽管可能仍然很大)。

\textbf{线性规划 LP1 (独立信号)}

\begin{align*}
    \max_{t,x} \quad & \sum_{\theta \in \Theta} \mu(\theta) t_{\theta} && (\text{目标:最大化期望收益}) \\
    \text{s.t.} \quad & \sum_{q \in Q^*} x_{\theta}(q) v_{\theta}(q) - t_{\theta} \geq \sum_{q \in Q^*} x_{\theta'}(q) v_{\theta}(q) - t_{\theta'} && \forall \theta, \theta' \in \Theta \quad (\text{IC:类型}\theta\text{不羡慕}\theta') \\
    & \sum_{q \in Q^*} x_{\theta}(q) v_{\theta}(q) - t_{\theta} \geq v_{\theta}(p) && \forall \theta \in \Theta \quad (\text{IR:购买不劣于不买}) \\
    & \sum_{q \in Q^*} x_{\theta}(q) \cdot q = p && \forall \theta \in \Theta \quad (\text{可行性:后验的期望是先验}) \\
    & \sum_{q \in Q^*} x_{\theta}(q) = 1 && \forall \theta \in \Theta \quad (\text{概率和为1}) \\
    & x_{\theta}(q) \geq 0, \quad t_{\theta} \geq 0 && \forall \theta, q \quad (\text{非负性})
\end{align*}

\begin{itemize}
    \item \textbf{IC 约束}:左边是类型为$\theta$的买方购买合约$(Y_\theta,t_\theta)$的净效用,右边是他冒充$\theta'$购买合约$(Y_{\theta'},t_{\theta'})$的净效用。注意,当他冒充时,他收到的是为$\theta'$设计的信号,但他评估价值时用的是自己的效用函数$v_\theta$。
    \item \textbf{IR 约束}:购买合约的净效用,不低于他的保留效用$v_\theta(p)$(基于先验$p$做决策的效用)。
    \item \textbf{可行性约束}:保证了信息披露的物理可能性。
\end{itemize}

这个LP的变量数量和约束数量都是关于$|\Theta|$和$|Q^*|$的多项式。虽然$|Q^*|$可能很大,但这个问题原则上是可以在多项式时间内求解的。更妙的是,通过考察其对偶问题 (Dual Problem),可以利用椭球法 (Ellipsoid Method) 和分离预言机 (Separation Oracle) 在多项式时间内求解,即使$|Q^*|$是指数级的。这证明了最优机制是可计算的。

\subsubsection{一个具体的计算例子}

让我们回顾第二章中的代码示例,并用这里的LP框架来分析它。

\begin{itemize}
    \item $\Theta=\{0,1\},\Omega=\{0,1\}$
    \item $v_0(q)$和$v_1(q)$是两个关于后验$q(\omega=1)$的分片线性凸函数。
    \item $p=(0.5,0.5)$是先验。
    \item 我们计算出$\xi(0)=2.0,\xi(1)=2.0$。保留效用$v_0(p)=v_1(p)=8.0$。
\end{itemize}

一个简单的可行解是:
\begin{itemize}
    \item \textbf{合约菜单}:只提供一个合约$(Y,t)$。
    \item \textbf{价格$t$}:设为2.0。
    \item \textbf{信息披露$Y$}:完全披露$\omega$。这等价于,如果$\omega=0$,生成的后验是$q=(1,0)$;如果$\omega=1$,生成的后验是$q=(0,1)$。
    \item \textbf{这个$Y$的可行性}:$0.5\times (1,0) + 0.5\times (0,1) = (0.5, 0.5)=p$,满足。
    \item \textbf{验证约束}:
       \begin{itemize}
         \item IR:对$\theta=0$,购买后的期望效用是10.0(因为总能做出正确决策),支付2.0,净效用8.0。不购买的保留效用也是8.0。所以$8.0 \geq 8.0$满足。$\theta=1$同理。
         \item IC:因为只有一个合约,IC约束自动满足。
       \end{itemize}
    \item \textbf{收益}:$\mu(0)\times t + \mu(1)\times t = 2.0$
\end{itemize}

这与我们之前通过"密封信封"机制找到的解是一致的。但LP框架更加通用,即使在$\xi(\theta)$不同,或者最优机制需要部分披露信息时,LP依然能找到最优解。

\subsection{引入预算约束:咨询机制的诞生}

引入一个关键的现实因素:预算约束。

每个买方$(\theta,b)$有一个预算$b$,他支付的总额不能超过$b$。为了简化,我们假设预算是公开的(即$\mu(\omega,\theta,b)$中,$b$是确定的)。私有预算的情况类似,只是增加了对$b$的激励约束。卖方也有一个预算$M$,她支付给买方的总额不能超过$M$。

\textbf{预算约束如何改变游戏?}它使得一种更简洁、更强大的机制成为可能——咨询机制 (Consulting Mechanism)。

\textbf{定义 4.1 (直接支付的咨询机制, CM-dirP)}:这是一种非常直观的机制,模拟了真实的咨询过程:

\begin{enumerate}
    \item \textbf{承诺} (Commit):卖方(顾问)首先公开承诺一套"服务方案"。对于每一种可能的买方报告类型$\theta$,她都定义了:
       \begin{itemize}
         \item 一个价格$t_\theta$。
         \item 一个推荐策略$p_\theta(a|\omega)$(当卖方观察到$\omega$时,她推荐行动$a$的概率)。
       \end{itemize}
    \item \textbf{报告} (Report):买方(客户)向卖方报告自己的类型$\theta$。
    \item \textbf{支付} (Pay):买方支付服务费$t_\theta$。
    \item \textbf{推荐} (Recommend):卖方根据真实的$\omega$和买方的报告$\theta$,按照$p_\theta(a|\omega)$随机抽取一个行动$a$,并将其作为"建议"告诉买方。
\end{enumerate}

\textbf{"听话"约束 (Obedience Constraint)}:这种机制有一个至关重要的内在约束,称为听话约束 (Obedience Constraint)。卖方承诺的推荐策略$p_\theta(a|\omega)$必须是"诚实"的,即当买方被推荐了行动$a$时,他自己计算后发现,$a$确实是当前情况下的最优选择。

形式化地,当买方被推荐行动$a$时,他会更新自己对$\omega$的信念。新的后验是$\Pr(\omega|\text{推荐}=a)$。他必须发现,最大化$\sum\limits_{\omega}\Pr(\omega | \text{推荐}=a)u(\theta,\omega,a')$的$a'$就是$a$。
$$a = \arg \max_{a' \in A}\sum\limits_{\omega\in\Omega}\frac{\mu(\omega)p_{\theta}(a|\omega)}{\sum_{\omega'}\mu(\omega')p_{\theta}(a|\omega')}u(\theta,\omega,a')$$

卖方需要设计的东西不再是抽象的"信号生成器",而是具体的"行动推荐策略"。由于行动集合$A$通常比后验空间$Q^*$小得多,这大大简化了设计空间。在有公开预算和独立信号的情况下,这种直接支付的咨询机制 (CM-dirP) 是最优的。

\subsubsection{咨询机制的线性规划 (LP)}

我们可以为 CM-dirP 构建一个更紧凑、变量更少的线性规划。

\textbf{决策变量:}
\begin{itemize}
    \item $t_\theta$:为类型$\theta$设计的价格。
    \item $p_\theta(a|\omega)$:推荐策略。为了让它成为LP的变量,我们通常定义$z_\theta(a,\omega)=\mu(\omega)p_\theta(a|\omega)$。$\sum\limits_{a}p_\theta(a|\omega)=1$的约束就变成了$\sum\limits_{a} z_\theta(a,\omega)=\mu(\omega)$。
\end{itemize}

\textbf{线性规划 LP2 (咨询机制)}

\begin{align*}
    \max_{t,z} \quad & \sum_{\theta \in \Theta} \mu(\theta) t_{\theta} \\
    \text{s.t.} \quad & \sum_{a, \omega} z_{\theta}(a, \omega) u(\theta, \omega, a) - t_{\theta} \geq \sum_{a, \omega} z_{\theta'}(a, \omega) u(\theta, \omega, a) - t_{\theta'} && \forall \theta, \theta' \quad (\text{IC}) \\
    & \sum_{a, \omega} z_{\theta}(a, \omega) u(\theta, \omega, a) - t_{\theta} \geq v_{\theta}(p) && \forall \theta \quad (\text{IR}) \\
    & \sum_{\omega} z_{\theta}(a, \omega) u(\theta, \omega, a) \geq \sum_{\omega} z_{\theta}(a, \omega) u(\theta, \omega, a') && \forall \theta, a, a' \quad (\text{Obedience}) \\
    & \sum_{a \in A} z_{\theta}(a, \omega) = \mu(\omega) && \forall \theta, \omega \quad (\text{可行性}) \\
    & t_{\theta} \leq b && \forall \theta \quad (\text{Budget}) \\
    & z_{\theta}(a, \omega) \geq 0, \quad t_{\theta} \geq 0 && \forall \theta, a, \omega
\end{align*}

这个LP的变量数量是$|\Theta|+|\Theta|\times |A|\times |\Omega|$,约束数量也是这些参数的多项式。这是一个标准大小的LP,可以用任何现成的求解器(如 Gurobi, CPLEX)高效求解。这比之前需要椭球法的 LP1 在实践中要高效得多。

\subsubsection{私有预算与存款-返还机制}

如果买方的预算$b$是其私有信息$\theta$的一部分,那么 CM-dirP 就不再最优了。因为买方可能会为了支付更低的服务费$t_\theta$而谎报自己的预算。

\textbf{存款-返还的咨询机制 (CM-depR)} 流程:

\begin{enumerate}
    \item \textbf{报告与存款}:买方报告其类型和预算$(\theta,b)$,并立即存入其声称的全部预算$b$作为押金。
    \item \textbf{推荐与返还}:卖方根据报告$(\theta,b)$和$\omega$,推荐一个行动$a$,并返还一笔金额给买方。最终买方支付的净额为$t_{\theta,b}$。净支付额$t_{\theta,b}$是由卖方设计的。返还的金额是$b-t_{\theta,b}$。
\end{enumerate}

"先存后返"的结构,使得买方无法通过谎报来获利。如果他低报预算,比如真实预算是100,他谎报是50,那么他最多只能存50,这会限制他能参与的合约。如果他高报预算,他根本拿不出那么多钱来做押金。这个简单的操作有效地"检验"了买方的预算声明,从而恢复了激励相容性。

CM-depR 是私有预算和独立信号下的最优机制,并且同样可以被一个标准大小的线性规划求解。

\subsection{终极挑战:相关信号与概率性返还}

当信号相关时,即使有预算约束,问题也变得更加复杂。买方的支付必须依赖于卖方最终揭示的信息。

一种被称为\textbf{概率性返还的咨询机制} (CM-probR) 在最一般的情况下(相关信号,私有预算)都是最优的。

\textbf{机制流程:}

\begin{enumerate}
    \item \textbf{报告与存款}:买方报告$(\theta,b)$并存入预算$b$。
    \item \textbf{推荐与概率性结果}:卖方根据$(\theta,b)$和$\omega$,随机决定一个组合$(a,i)$,其中$a$是推荐行动,$i$是一个支付结果的指示器,$i \in \{\text{返还}0, \text{返还}(b+M)\}$。
    \item 如果$i$是"返还0",买方净支付$b$。如果$i$是"返还$(b+M)$",卖方向买方支付$M$,买方净支付$-M$。
\end{enumerate}

这个机制看起来非常奇特,为什么要引入卖方预算$M$,并且支付额如此极端?

\textbf{强烈的激励信号:}这两种极端的支付结果,向不同类型的买方传递了非常强烈的不同信号。因为信号是相关的,类型为$\theta$和$\theta'$的买方对$\omega$的信念不同,因此他们对"最终支付$b$"和"最终倒拿$M$"这两种情况发生的概率预期也不同。

\textbf{利用相关性:}卖方正是利用了这种预期的差异,来精细地进行价格歧视,从而榨取最大收益。这呼应了著名的 Cremer-McLean 机制的思想,即当买方类型充分相关时,卖方几乎可以榨取所有剩余价值。